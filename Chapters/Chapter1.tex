\chapter{Introducción General}

\label{Chapter1} % \ref{Chapter1} 
\label{IntroGeneral}

%-------------------------------------------------------------------------------

\newcommand{\keyword}[1]{\textbf{#1}}
\newcommand{\tabhead}[1]{\textbf{#1}}
\newcommand{\code}[1]{\texttt{#1}}
\newcommand{\file}[1]{\texttt{\bfseries#1}}
\newcommand{\option}[1]{\texttt{\itshape#1}}
\newcommand{\grados}{$^{\circ}$}

%-------------------------------------------------------------------------------

\paragraph{} En este capítulo se presentan la necesidades generales de la industria y las particulares del cliente que se pretenden satisfacer. Adicionalmente, se provee de una introducción técnica superficial de las tecnologías involucradas.

\section{Motivación}
\label{motivacion}

\subsection{Necesidades generales}

	\subsubsection{Progreso tecnológico}

	\paragraph{} La producción de bienes y servicios experimentó una serie de innovaciones que fueron reemplazando distintas habilidades y características del hombre. Se identifican cuatro hitos en este proceso, los cuales se denominan revoluciones industriales.
	\paragraph{} La primer revolución industrial consistió en reemplazar las fuerzas físicas de hombres y bestias en favor de nuevas fuentes de energía para obtener trabajo. Vapor, combustión, electricidad, ect. 
	\paragraph{} La segunda revolución industrial hizo obsoleta la capacidad de polivalencia de los humanos al crear líneas de montaje, gracias a Henry Ford nació la organización productiva moderna.
	\paragraph{} Un tercer hito fue la llegada de la electrónica y ordenadores que permitió crear nuevas tecnologías de instrumentación y control, esto permitió reemplazar la motricidad gruesa y fina de las personas, también desplazó las tareas manuales de control de actuadores. 
	\paragraph{}Finalmente, la cuarta y actual revolución industrial consiste en el reemplazo de las capacidades cognitivas biológicas al conectar los dispositivos entre sí y con ordenadores que proveen persistencia en las mediciones y su análisis a través de la inteligencia artificial.
	
	\subsubsection{Modernización}

		\paragraph{} Las plantas de producción que se encuentran en la República Argentina están retrasadas en su progreso tecnológico, la mayoría no ha incorporado sistemas electrónicos en sus procesos o productos. Esto plantea la necesidad de crear un sistema que logre adaptar la antigua tecnología existente y que permita integrarla a los nuevos paradigmas de trabajo. Finalmente, la actual situación tecnológica, se traduce en la demanda un sistema que permita unir capital viejo y sus protocolos antiguos con las nuevas posibilidades que ofrece la cuarta revolución industrial.

	\subsubsection{Comercio internacional}
		\paragraph{} El avance tecnológico produce cambios en el marco normativo de las naciones, la consecuencia es la necesidad de conformar con requisitos mínimos de capital para cumplir con los requerimientos jurídicos necesarios para ingresar dentro de nuevos mercados. El retraso tecnológico ya no solo genera una pérdida de competitividad, sino que también implica una barrera legal que impide que las empresas dentro de la jurisdicción argentina coloquen sus mercancías en otras latitudes. Por incumplimiento en normas que apuntan a la calidad de la producción o a la protección del medio ambiente.
		\paragraph{} Las necesidades generales expuestas promovieron el interés de crear un proyecto que permita asistir a las empresas en su incorporación a la cuarta revolución industrial. Si las compañías argentinas no lo lograran, probablemente queden fuera del mercado mundial en pocos años.

\subsection{Necesidades del cliente}
	\subsubsection{Gador S.A}
		\paragraph{} \emph{Gador S.A} es una compañía cuya misión es \emph{producir medicamentos para la salud humana, con la mejor calidad disponible, y ponerlos al alcance de la comunidad a precios accesibles}. 
		% Tecnología de Gador.
		\paragraph{} La arquitectura de monitoreo ambiental dentro de sus plantas se encuentra gestionada por el producto \emph{Enterprise Buildings Integrator (EBI)}, de la empresa \emph{Honeywell}.
% el cliente
% atrazo tecnológico
% normas de exportación



\section{Introducción técnica}
\label{introTecnica}

\section{Estado del arte}
\label{estadoArte}

\section{Objetivos y alcance}
\label{objetivos}
